% \documentclass[compress,12pt]{beamer}
\documentclass[slidestop,compress,12pt]{beamer}
%下面是设置主题的
% \usetheme{Warsaw}
\usetheme{AnnArbor} 
% \usetheme{CambridgeUS}
% \usetheme{JuanLesPins}
% \usecolortheme{whale}
% \usecolortheme{dove}
\usecolortheme{crane}
%设置item前面的小箭头的样式的
%\useinnertheme{default}
\useinnertheme{circles}
\usepackage{color}  %颜色
\usepackage{amsmath,amssymb,latexsym}  %数学
\usepackage{wasysym}
\usepackage{mathrsfs}
\usepackage{graphicx}
\usepackage{caption}
\usepackage{bm}
\usepackage{amsmath,amssymb}
\usepackage[export]{adjustbox}

\newcommand{\citepaper}[1]{\color{black}{\scriptsize{$\sun$~\it{#1}}}}
% \usepackage{beamerthemesplit} // Activate for custom appearance

\DeclareMathOperator{\diff}{d\!}
\DeclareMathOperator{\Diff}{D\!}
\newcommand{\im}{\mathrm{i}}
\newcommand{\ee}{\mathrm{e}}

\setbeamertemplate{caption}{\raggedright\insertcaption\par}

\title{Electron Tomography and Its Limitations}
\author{Xin Chen}
\date{\today}

\begin{document}
\def\mean#1{\left< #1 \right>}
\frame{\titlepage}

\section[Existing works]{}

\begin{frame}
    \begin{figure}
        \includegraphics[width=12cm]{imgs/eg1.png}
        \caption{Grünewald, Kay, et al, 2003, resolution: 7nm}
    \end{figure}
\end{frame}

\begin{frame}
    \begin{figure}
        \includegraphics[width=11cm]{imgs/eg2.png}
        \caption{Lei Zhang, et al, 2016, resolution: 17.1\AA, dsDNA:  30nm, gold particle: 6nm}
    \end{figure}
\end{frame}

\section[Radon Transform]{}
\begin{frame}
    \begin{minipage}[c]{0.4\textwidth}
        \begin{figure}
        \includegraphics[scale=0.4, left]{imgs/radon1.png}
        \end{figure}
    \end{minipage}%%%
    \begin{minipage}{0.5\textwidth}
    \scriptsize
        \begin{equation}
            \begin{split}
                p(t, \varphi) &= \int\int f(x,y) \delta (x \cos{\varphi} + y \sin{\varphi} - t) \diff x \diff y \notag \\
                P(k, \varphi) &= \text{FT}1[p] = \int\int f(x,y) \ee ^ {-\im k x\cos{\varphi}  - \im k y\sin{\varphi} } \diff x \diff y \\
                & = \int\int f(x,y) \ee ^ {-\im ux  - \im vk} \diff x \diff y \\
                u &= k\cos{\varphi} \\
                v &= k\sin{\varphi}
            \end{split}
        \end{equation}
    \end{minipage}%

    Fourier Slice Theorem: 1D FT of $p(t,\varphi)$ is equal to the slice of 2D FT of $f(x,y)$ at the angle $\phi$ --- the math for iRadon.
    \vspace{-0.23cm}
    \begin{figure}
        \includegraphics[scale=0.15]{imgs/iradon.jpg}
    \end{figure}

\end{frame}

\begin{frame}{2D Reconstruction}
    \begin{figure}
        \includegraphics[scale=0.35]{imgs/eg3.png}
    \end{figure}
\end{frame}

\section[3D Reconstruction]{}
\begin{frame}{3D Reconstruction}
    \begin{figure}
    \centering
        \begin{minipage}{0.2\textwidth}
            \includegraphics[width=1\linewidth]{imgs/obj.png}
        \end{minipage}%
        \begin{minipage}{0.8\textwidth}
            \includegraphics[width=1\linewidth]{imgs/mid_slices.png}
        \end{minipage}
    \end{figure}
\end{frame}

\begin{frame}
    \begin{figure}
        \includegraphics[scale=0.5,left]{imgs/sinogram.png}
    \end{figure}\pause
    \vspace{-3cm}
    \begin{figure}
        \includegraphics[scale=0.5,right]{imgs/projections.png}
    \end{figure}
\end{frame}

\section{Validity of TEM tomography}

\begin{frame}{Image delivered projection}
    \vspace{-0.5cm}
    \begin{figure}
        \includegraphics[scale=0.2]{imgs/TEM_black_box.png}
    \end{figure}
    \vspace{-0.5cm}
    The simplest model: Projection Approx + Weak Phase Object Approx
    \begin{equation}
        % \begin{split}
            % \psi_{i}^{BF}(\bm{r}) = [1 + \im\sigma_e \nu_{\text{proj}}(\bm{r})] \circledast \text{FT}^{-1} [\ee^{-\im \chi (\bm{k})}] \notag \\
            \Psi_{i}^{\text{BF}}(\bm{k}) = [\delta(0) + \im\sigma_e \mathcal{V}_{\text{proj}}(\bm{k})] \ee^{-\im \chi (\bm{k})} \notag 
        % \end{split}
    \end{equation}
    Tune $\chi$ to achieve $-\pi/2$ phase shift.
    \begin{equation}
        \psi^{\text{BF}}_i (\bm{r}) = 1-\sigma_e \nu(\bm{r}) ,\ 
        g^{\text{BF}}(\bm{r}) = 1-2\sigma_e\nu_{\text{proj}} \notag
    \end{equation}
    ADF-TEM: transmitted wave is excluded.
    \begin{equation}
        \psi^{\text{ADF}}_i(\bm{r}) = 1 - \psi^{\text{BF}}_i(\bm{r}) ,\ 
        g^{\text{ADF}}(\bm{r}) = [\sigma_e \nu(\bm{r})]^2 \notag
    \end{equation}
\end{frame}

\begin{frame}{Validity}
    \begin{itemize}
    \item PA: single scattering, thin specimen, potential varies slowly over distance $\sqrt{\Delta z \lambda / 2\pi}$ (Kazuo Ishizuka \& Natsu Uyeda, 1977). And the spatial frequency $q$ of interest is much smaller than $\sqrt{1/4\lambda \Delta z}$ (cited by Vulović, Miloš, et al, 2014).
    \item WPOA: thin specimen, low atomic number, $\sigma_e \nu < 0.36$ (Vulović, Miloš, et al, 2014).
    \end{itemize}
    \vspace{-1cm}
    \begin{figure}
        \includegraphics[scale=0.3, right]{imgs/validity.png}
    \end{figure}
\end{frame}

\begin{frame}{Another problem: electron does}

Tow requirements that contradict each other (Wolfgang Baumeister, Rudo Grimmand Jochen Walz, 1999):
    \begin{itemize}
        \item minimum electron does.
        \item sampling as much as possible and high SNR
    \end{itemize}
\vspace{0.5cm}
    commonly used does: $10-20 \text{e}^{-1}/$\AA${}^2$
    % \begin{figure}
    %     \includegraphics[scale=0.4, right]{imgs/SNR_does.png}
    % \end{figure}
\end{frame}

\begin{frame}[c]{A numerical approach to study the limitations}
    \begin{figure}
        \includegraphics[scale=0.5]{imgs/project.png}
    \end{figure}
\end{frame}

\begin{frame}[c]
\centering
\huge
\textbf{Thanks!}
\end{frame}
\end{document}
























